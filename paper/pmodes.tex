\documentclass[12pt, letterpaper]{article}

\setlength{\oddsidemargin}{0.25in}
\setlength{\textwidth}{6.00in}
\setlength{\topmargin}{-0.30in}
\setlength{\textheight}{9.60in}

\begin{document}\sloppy\sloppypar\raggedbottom\frenchspacing

\section*{\raggedright Spectral identification and nulling of stellar oscillations in extremely precise radial-velocity experiments}

\noindent
Zhao, Bedell, Hogg, and others

\paragraph{abstract}
% context
Extremely precise radial-velocity (RV) measurements are contaminated
by asteroseismic p-mode oscillations, because such oscillations
produce surface motions.
Such oscillations are associated with stellar radius and shape
changes, and therefore brightness and temperature variations, with
informative relationships between these and the contaminating
radial-velocity signals they produce.
% aims
Here we attempt to measure, spectrally, the temperature variations
associated with p-modes, and use them to calibrate out or clean RV
measurements of these modes.
% methods
We fit data from the EXPRES and HARPS instruments using a
time-independent stellar spectral model, and then project the
residuals in the spectral space onto spectral and temporal templates.
% results
We find that we can see temperature variations at the XXX K LEVEL.
We find...OTHER THINGS.
% punchline
The temperature measurements and p-mode mitigation can be made with
exactly the same data and instrument that measures the RVs; they do
not require additional data.
However, our findings suggest that RV experiments are better off
taking more but shorter exposures at each stellar visit.

\section{Introduction}

This has been done before (sort-of) with brightness variations. That
sucks for REASONS.

\section{Method}

We run wobble. We get residuals.

...

\end{document}
