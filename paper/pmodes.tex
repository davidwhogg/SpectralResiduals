\documentclass[12pt, letterpaper]{article}

\setlength{\oddsidemargin}{0.25in}
\setlength{\textwidth}{6.00in}
\setlength{\topmargin}{-0.40in}
\setlength{\textheight}{8.70in}

\begin{document}\sloppy\sloppypar\raggedbottom\frenchspacing

\section*{\raggedright Spectral identification and nulling of stellar oscillations in extremely precise radial-velocity experiments}

\noindent
Zhao, Bedell, Hogg, and others

\paragraph{abstract}
% context
Extremely precise radial-velocity (RV) measurements are contaminated
by asteroseismic p-mode oscillations, because such oscillations
produce surface motions.
Such oscillations are associated with stellar radius and shape
changes, and therefore brightness and temperature variations, with
informative relationships between these and the contaminating
radial-velocity signals they produce.
% aims
Here we attempt to measure, spectrally, the temperature variations
associated with p-modes, and use them to calibrate out or clean RV
measurements of these modes.
% methods
We fit data from the EXPRES and HARPS instruments using a
time-independent stellar spectral model, and then project the
residuals in the spectral space onto spectral and temporal templates.
% results
We find that we can see temperature variations at the XXX K LEVEL.
We find...OTHER THINGS.
% punchline
The temperature measurements and p-mode mitigation can be made with
exactly the same data and instrument that measures the RVs; they do
not require additional data.
However, our findings suggest that RV experiments are better off
taking more but shorter exposures at each stellar visit.

\section{Introduction}

This has been done before (sort-of) with brightness variations. That
sucks for REASONS.

\section{Method}

We run wobble. Wobble makes the assumption that the star never varies,
except in RV. Wobble gives us a set of RV measurements $v_n$, one for
each observation epoch $t_n$.

We fourier transform (or perform some equivalent operation on) these
RV measurements to find near-periodic oscillation modes with
frequencies $\nu_k$. We show that these are plausibly p modes,
observed in RV.

Reminder, wobble works in log flux; that is, the data $y_n$ on
spectrum $n$ are natural logarithms of continuum-normalized flux
values.

The wobble model is generative, so in addition to the RVs $v_n$ we
also get residuals in the space of the data. That is, for every epoch
$t_n$, we get a difference $\Delta y_n$ between the data $y_n$ at that
epoch and the best-fit prediction of the wobble model.

These residuals are in the data space, which is a (vacuum, log)
wavelength basis that is at rest with respect to the observatory. This
means that the spectrum $y_n$ has has had subtracted from it the
best-fit constant template spectrum, shifted to the best-fit Doppler
Shift under the assumption that there is no spectral variability.

We seek to find the spectral variability associated with p-mode
oscillations in these residuals. We have (more than) two options.

\bigskip\noindent
\textsl{Spectral-expectation option}---In this setting, we take the residuals
$\Delta y_n$ at each epoch $n$ and project them onto the derivative of
the spectral (log flux) expectation with respect to (log)
temperature. These projections ought to be done at the best-fit RV
relative to the observatory rest frame (the wobble-output RV). That
is, we need to shift the spectral expectation derivatives before
projection. These projections can be processed into temperature
offsets $\Delta T_n$, one per epoch $t_n$..

We show that these temperature offsets $\Delta T_n$ show fourier modes
that are very similar in frequencies to the RV modes.

Bonus points: We show that we can measure stellar temperature variations
with incredible (formal) precision.

Extra bonus points: We show phase relationships and that the
temperature observations can be used to predict or correct the
RVs. That is, we find a mitigation strategy. It probably involves the
temperature variation having some approximately-$2\pi$ phase offset
with respect to RV variation, depending on $\ell,m$.

All this depends on having a form for the derivative of the spectral
expectation with respect to temperature. This can be obtained
data-driven using stellar twins (and I have both trivial and
sophisticated ideas there) or it can be obtained by exercising a
stellar atmosphere model. It doesn't matter in practice how good this
is, because the project only depends on there being a good dot product
between what we use and the true derivative.

\bigskip\noindent
\textsl{Time-expectation option}---We take the residuals $\Delta y_n$ and weight
them with sines and cosines of $2\pi\,\nu_k\,t_n$. We then shift to a
common stellar rest frame (using the wobble output RVs $v_n$) and
construct weighted average spectral residuals for sine and cosine for
each frequency $\nu_k$.  These weighted averages are the spectral
variations that are coherent with each normal mode $k$.

We show that these spectral variations that are coherent with each
mode $k$ corresponds (at least roughly) to temperature variations.

Bonus points: We show that the phase relationships between the
spectral and RV variations make sense in terms of a physical model of
the stellar surface.

Extra bonus points: We find a mitigation strategy.


...

\end{document}
